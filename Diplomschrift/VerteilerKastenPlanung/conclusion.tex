\chapter{Conclusion}

This document is the first of two parts of the thesis, the second part is scheduled for publication by my colleague in September 2020. Although he has already done significant and important fundamental research such as a detailed software evaluation and testing of Wordpress, these are not included in this document due to time related issues. Both parts of the work are based on the specified criteria and definitions which we were able to determine along with our Erasmus representative in \ref{target_condition}. It was essential to understand the difference between a virtual and traditional museum in order to achieve a common consensus of conceptions. Because of these specified concepts and constraints, my colleague was able to determine, based on a in-depth software evaluation including more than seven alternative approaches, that either Wordpress or Omeka S are the most promising solutions to realise the proposed virtual museum.

While my colleague examined Wordpress in greater detail, we created a local test museum with Omeka using modern virtualisation techniques. Firstly, we investigated and proposed a possible architecture for a virtual museum within Omeka (see \ref{museum_arch}), followed by the installation (see \ref{install}), compared user privileges and roles to the requirements (see \ref{omeka_users_permissions}), tested user creation and management (see \ref{manage_users}), checked language support (see \ref{langs}), added exhibits using metadata standards and media (see \ref{add_exhibits}), tested how to build and link sub-museums (see \ref{add_museum}), researched the extent Omeka is capable of supporting themes and modules and their range of functionality (see \ref{omeka_themes_modules}), created individual pages (see \ref{add_pages}), checked the capabilities of their RESTful API (see \ref{omeka_api}), embedded and developed maps of sub-museums (see \ref{omeka_map_of_museums}), and evaluated the extent to which Omeka allows interaction for visitors and its limitations (see \ref{omeka_interactivity}). The test of the toolkit was completed by designing and setting up a test museum (see \ref{omeka_room_exp}), whereby it was attempted to adhere to all defined specifications while keeping the development effort to a minimum. Although we were able to create a virtual museum with two sub-museums, and almost all of the specified requirements were met, the test failed. To be able to implement most of the interactivity, like the map of museums or the graphical navigation, it was essential to embed those features using self-developed JavaScript codes. These solutions are difficult to maintain and there exist no user interfaces to build, for instance, rooms of exhibits. A separate module would have to be developed for Omeka, integrating essential key features mentioned in \ref{omeka_conc}. Despite the fact that software developers are part of the final implementation team, an explicit requirement was to keep the coding effort as low as possible. Nevertheless, we have written a detailed multilingual documentation for future students, teachers and administrators in case that Omeka is used anyway as a toolkit to realise the project. 

During an objective comparison of the two toolkits afterwards, Wordpress convinced us due to the large number of existing plugins. The development workload can be significantly reduced by this vast number of available plugins and extensions, since almost all features and required functionalities of \ref{target_condition} can be embedded rapidly and efficiently. Although Wordpress also requires a custom plugin, it only serves to link existing plugins and manage simple data structures within the database. Consequently, we decided to use Wordpress in further development and recommend it for the final implementation of the virtual museum.


Based on the testing result, we researched to what extent Wordpress supports the integration of social media and the linking with other virtual museums in \ref{cha:Integration of Planned Features}. We planned a total of four use cases, regarding the integration of social networks while discussing and suggesting possibilities of implementation for each. It turns out, Wordpress already provides developed plugins in the majority of cases, which saves future developers a considerable amount of work. In order to integrate other museums, we first had to seek partners and possible institutions that would be interested in a collaborative approach. In this matter, we particularly noticed the EULAC-MUSEUMS project, which was financed by the European Union like our project. Wordpress offers in their official plugin repository a variety of plugins that allow us to embed resources from, for example, EULAC-MUSEUMS.

In the context of this thesis, a total of 2 virtual machines with Linux distributions were set up, which represent complete realisation prototypes for the virtual museum. Both would be ready for immediate use and include bilingual documentation for students, teachers and administrators to add and manage exhibits and exhibition spaces.

Based on this thesis, development teams of the HTLBLuVA Wiener Neustadt (HTLWRN) are going to develop a virtual museum for the project in the following years. The extent to which this document will be followed and the relevance of the analyses and tests performed depends heavily on my colleague's publication in September. In any case, by the means of the temporarily created Omeka museum and written documentation, partner institutions are already able to add exhibits which can later be transferred to the proper museum.
