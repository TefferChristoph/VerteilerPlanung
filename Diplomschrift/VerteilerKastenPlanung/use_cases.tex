\chapter{Test Cases (FM)}
\label{cha:Test Cases}

\section{Start}

On the one hand, we have to teach our users how to add exhibits to the museum. On the other hand, we have to teach them how to use the museum properly and install any tools they may need. Ideally, the platform should be completely self-explanatory. The documentation should also be written in English, as the museum is intended for international use.

\section{Documentation}

We can assume that the majority of the users of our virtual museum will neither be tech-savvy nor will they be interested in learning how to fill the museum with content. We also have to take into consideration that many users may be very young, as schools are also involved in the Erasmus project. That is why the approach we are adopting in teaching users how to add exhibits is very important and is covered in detail in this chapter.

We are convinced that documentation should exist in some form. Different formats of documentation offer different advantages and disadvantages, which should be evaluated first.

\subsection{Technical LATEX or Markdown  Documenation}

A common way (especially in the technical and scientific field) to write documentation for the use of tools are Markdown and LATEX. Where LATEX is used more in science and Markdown more in software development. You can get a large amount of information in a relatively neat and clear form. Also, different formatting elements like tables, images, code and mathematical formulas offer many possibilities to display many types of content well. 

\vspace{0.5cm}

Another advantage is the low memory consumption in contrast to other forms of documentation. For users with a poor Internet connection, this is the most sensible way to quickly get an effective introduction and support. All the user will need is a program with which they are able to read PDF files. Typically, every common operating system has capabilities to open them.

\vspace{0.5cm}

Despite all the advantages of this form of documentation, we will not choose it as our main documentation. Such documents usually contain many pages that are overloaded with text and rather boring to read.  For some users this might be a useful form of documentation, but not for our target group. Especially younger users will have a hard time reading this kind of documentation and working through it extensively.

\vspace{0.5cm}

We are more likely to create a short reference manual in this form with which users can quickly find support. But not to get an introduction.

\subsection{Knowledge Base}

A knowledge base can be found on many websites nowadays and is an alternative to the Markdown and Latex documentation. Instead of one big document, many relatively small articles are created and run like a kind of blog where these posts can be categorized and searched. Typically Markdown is also used to create these articles.

\subsection{Videos}

Videos are very common when it comes to teaching things and providing introductions. If they are well-designed, they reduce the effort for us and help users immensely, because they see everything straight away. Through platforms like YouTube, we are also able to publish and embed these videos completely free of charge. Indirectly, they also represent a good advertising opportunity. Especially the younger generation has grown up with videos and will have less problems to accept this medium than a long documentary.

\vspace{0.5cm}

The biggest disadvantage is probably the amount of data on this medium. Especially schools are known to have a miserable internet connection. Therefore, these videos should also be made available as an offline package.


\section{Test Cases}

The main purpose of this documentation is to provide an introduction for users and to explain how to use different aspects of the toolkit.

\subsection{Sign In for Students, Lecturers, and Administrators}

This aspect of the documentation will make sense in all proposed forms. In the form of a video there should be a short step-by-step guide showing how the user can add exhibits. There should also be an explanation of how to install the necessary software to use the toolkit and how users can sign in with their access data. If it is described in a knowledge base, the whole documentation can easily be divided into several articles that reference each other. In a Markdown/Latex documentation, the PDF can be delivered together with the credentials, which makes it quick and easy for the user.

\subsection{Create Student Accounts Lecturer Student}